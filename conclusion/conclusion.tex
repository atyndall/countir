\documentclass[../thesis/thesis.tex]{subfiles}
\begin{document}
 \chapter{Conclusions}
% What did you do?
% Why did you do it?
% What happened?
% What do the results mean?
% What is your work good for?

The smart-home economy continues to grow, with automation being one of the main areas driving growth. The ability to detect people present within a space is an important smart-automation feature, with the implications for climate control energy efficiency alone being highly significant.

This project has attempted to create an occupancy detection system for such a smart home environment that meets four criteria; Low Cost, Non-Invasive, Energy Efficient and Reliable. Building such a system to commercial standards is outside of the scope of this project, however a prototype that attempts to prove the concepts involved was built and tested against these criteria. This prototype was based upon the ceiling-mounted thermal imaging approach of Thermosense \cite{beltran2013thermosense}, which after extensive analysis proved to be the best option given our criteria.

This prototype both validates the methods and results of the Thermosense paper, discovers key caveats surrounding the Thermosense approach, and also creates a software and hardware base on which future research into the area of occupancy in thermal imaging can be explored.

\section{Accuracy}
TODO: Summarize results.

As discussed in the Results section, the prototype developed achieves excellent reliability, citing accuracies in the 80\% range. 

\section{Energy Efficiency}
TODO: Summarize 4.3

\section{Privacy}
% TODO: Reference specific section of lit review
% TODO: Is privacy the same as non-invasive?
As discussed in the Literature Review, low-resolution thermal sensing provides the best trade-off between accuracy and invasiveness. Due to sensing in the infrared spectrum, it becomes significantly harder to surveil people in a malicious way, as many identifying features of people are not visible in the IR spectrum. This is compounded by the low resolution, which similarly assists in reducing the invasiveness of the sensor.

\section{Cost}
TODO: Summarize findings about cost

\section{Future Directions}
This project merely touched upon the area of thermal sensing and occupancy detection, and has laid the foundation for many more projects that build upon this original project. Some areas of future research are discussed here;

\subsection{Sub-pixel localization}
Due to the overlapping bell-curve characteristics of the \mlx's pixels, it may be possible to perform sub-pixel localization on objects within images. % TODO: Expand

\subsection{Improving Robustness}
One of the main areas of the project that was not explored due to time was the introducting of a wireless mesh networking architecture to the project. Future prototypes would consist of an many-to-one relationship between the Sensing/Pre-processing tier and the Analysis tier. Exploring the best way to mesh network these components while maintaining all the pre-existing criteria of the project would be challenging. In Appendix \Fref{chap:architecture} we provide our thoughts on the potential structure this could take.

Similarly, the current prototype uses a breadboarded structure that increases the size of the prototype significantly, as well as reduces the reliability of the prototype in the long-term. Converting the \mlx and PIR into a printed circuit board that fits onto the Arduino as a shield would both reduce the size of the prototype, as well as improving reliability for the future.

\subsection{Field-of-view modifications}
Several different techniques could be used to improve upon the field-of-view limitations of the \mlx, and exploring them and their cost/complexity implications would be useful. The first of these is applying a lens to the sensor, effectively expanding the field-of-view, but at the cost of distorting the image. Compensating for this distortion while maintaining accuracy presents an intriguing problem.

In another direction, using a motor with the \mlx to ``sweep'' the room, and thereby constructing a larger image of the space could also resolve the field-of-view issues. However, this approach also presents problems in stitching the images together in a sensible way, the distortion caused by rotating the sensor, as well as handing cases in which a fast-moving object is represented multiple times in the stitched image.

\subsection{New Sensors}
During this project, an updated version of our sensor, the MLX90621, was released. This version doubles the field-of-view in both the horizontal and vertical directions, addressing many of the problems encountered with the size of detection area in low-ceiling rooms. This version offers nearly complete backwards compatibility with the older version. Updating the project code-base to support it and re-running the experiments with the increased field-of-view to determine how much of an improvement it is would be interesting.

In addition to this, significantly higher resolution sensors are beginning to come to the market. The FLiR Lepton \cite{flir}, which sells in a dev kit for \$350, offers an $80 \times 60$ pixel sensor with a comparable field-of-view to the Grid-EYE. Exploring the increases in accuracy achievable though such significant increases in resolution would have significant contrion.
 
 \ifcsdef{mainfile}{}{\bibliography{../references/primary}}
\end{document}
