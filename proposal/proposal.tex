\documentclass[12pt, a4paper]{article}
\bibliographystyle{acm}
\usepackage{url}

\setlength{\oddsidemargin}{0.5cm}
\setlength{\evensidemargin}{0.5cm}
\setlength{\topmargin}{-1.6cm}
\setlength{\leftmargin}{0.5cm}
\setlength{\rightmargin}{0.5cm}
\setlength{\textheight}{24.00cm} 
\setlength{\textwidth}{15.00cm}
\parindent 0pt
\parskip 5pt
\pagestyle{plain}

\title{Research Proposal}
\author{}
\date{}

\newcommand{\namelistlabel}[1]{\mbox{#1}\hfil}
\newenvironment{namelist}[1]{%1
\begin{list}{}
    {
        \let\makelabel\namelistlabel
        \settowidth{\labelwidth}{#1}
        \setlength{\leftmargin}{1.1\labelwidth}
    }
  }{%1
\end{list}}

\begin{document}
\maketitle

\begin{namelist}{xxxxxxxxxxxx}
\item[{\bf Title:}]
	Developing a robust system for occupancy detection in the household
\item[{\bf Author:}]
	Ash Tyndall
\item[{\bf Supervisor:}]
	Professor Rachel Cardell-Oliver
\item[{\bf Degree:}]
	BCompSci (24 point project)
\item[{\bf Date:}]
	\today
\end{namelist}

\section*{Background} 
% In this section you should give some background to your
% research area. What is the problem you are tackling, and why is it
% worthwhile solving? Who has already done some work in this area,
% and what have they achieved?

The proportion of elderly and mobility-impaired people is predicted to grow dramatically over the next century, leaving a large proportion of the population unable to care for themselves, and consequently less people able care for these groups. \cite{chan2009smart} With this issue looming, investments are being made into a variety of technologies that can provide the support these groups need to live independent of human assistance. 

With recent advancements in low cost embedded computing, such as the Arduino \cite{Ardunio} and Raspberry Pi, \cite{RPi} the ability to provide a set of interconnected sensors, actuators and interfaces to enable a low-cost `smart home for the disabled' is becoming increasingly achievable.

Sensing techniques to determine occupancy, the detection of the presence and number of people in an area, are of particular use to the elderly and disabled. Detection can be used to inform various devices that change state depending on the user's location, including the better regulation energy hungry devices to help reduce financial burden. Household climate control, which in some regions of Australia accounts for up to 40\% of energy usage \cite{abs4602} is one particular area in which occupancy detection can reduce costs, as efficiency can be increased dramatically with annual energy savings of up to 25\% found in some cases. \cite{erickson2013thermosense}

Significant research has been performed into the occupancy field, with a focus on improving the energy efficiency of both office buildings and households. This is achieved through a variety of sensing means, including thermal arrays, \cite{beltran2013thermosense} ultrasonic sensors, \cite{hnat2012doorjamb} smart phone tracking, \cite{kleiminger2013using}\cite{balaji2013sentinel} electricity consumption, \cite{kleiminger2013occupancy} network traffic analysis, \cite{ting2013occupancy} sound, \cite{hailemariam2011real} CO2, \cite{hailemariam2011real} passive infrared, \cite{hailemariam2011real} video cameras, \cite{erickson2013poem} and various fusions of the above. \cite{yang2012multi}\cite{ting2013occupancy}


\section*{Aim}
% Now state explicitly the hypothesis you aim to
% test. Make references to the items listed in the Reference section
% that back up your arguments for why this is a reasonable
% hypothesis to test, for example the work of Knuth.
% Explain what you expect will be accomplished by undertaking this
% particular project.  Moreover, is it likely to have any other
% applications?

While many of the above solutions achieve excellent accuracies, in many cases they suffer from problems of installation logistics, difficult assembly, assumptions on user's technology ownership and component cost. In a smart home for the disabled, accuracy is important, but accessibility is paramount.

The goal of this research project is to devise an occupancy detection system that forms part of a larger `smart home for the disabled' that meets the following accessibility criteria;

\begin{itemize}
 \item \emph{Low Cost}: The set of components required should aim to minimise cost, as these devices are intended to be deployed in situations where the serviced user may be financially restricted.
 
 \item \emph{Non-Invasive}: The sensors used in the system should gather as little information as necessary to achieve the detection goal; there are privacy concerns with the use of high-definition sensors.
 
 \item \emph{Energy Efficient}: The system may be placed in a location where there is no access to mains power (i.e. roof), and the retrofitting of appropriate power can be difficult; the ability to survive for long periods on only battery power is advantageous.
 
 \item \emph{Reliable}: The system should be able to operate without user intervention or frequent maintenance, and should be able to perform its occupancy detection goal with a high degree of accuracy.
\end{itemize}

Success in this project would involve both
\begin{enumerate}
 \item Devising a bill of materials that can be purchased off-the-shelf, assembled without difficulty, on which a software platform can be installed that performs analysis of the sensor data and provides a simple answer to the occupancy question, and
 \item Using those materials and softwares to create a final demonstration prototype whose success can be tested in controlled and real-world conditions.
\end{enumerate}

This system would be extensible, based on open standards such as REST or CoAP, \cite{guinard2012search}\cite{kovatsch2013coap} and could easily fit into a larger `smart home for the disabled' or internet-of-things system.

 
\section*{Method}
% In this section you should outline how you intend to go
% about accomplishing the aims you have set in the previous
% section. Try to break your grand aims down into small,
% achievable tasks. Try to estimate how long you will
% spend on each task, and draw up a timetable for each
% sub-task.

Achieving these aims involves performing research and development in several discrete phases.

\subsection*{Hardware}
A list of possible sensor candidates will be developed, and these candidates will be ranked according to their adherence to the four accessibility criteria outlined above. Primarily the sensor ranking will consider the cost, invasiveness and reliability of detection, as the sensors themselves do not form a large part of the power requirement.

Similarly, a list of possible embedded boards to act as the sensor's host and data analysis platform will be created. Primarily, they will be ranked on cost, energy efficiency and reliability of programming/system stability.

Low-powered wireless protocols will also be investigated, to determine which is most suitable for the device; providing enough range at low power consumption to allow easy and reliable communication with the hardware.

Once promising candidates have been identified, components will be purchased and analysed to determine how well they can integrate.

\subsection*{Classification}
Depending on the final sensor choice, relevant experiments will be performed to determine the classification algorithm with the best occupancy determination accuracy. This will involve the deployment of a prototype to perform data gathering, as well as another device/person to assess ground truth.

\subsection*{Robustness / API}
Once the classification algorithm and hardware are finalised, an easy to use API will be developed to allow the data the device collects to be integrated into a broader system.

The finalised product will be architected into a easy-to-install software solution that will allow someone without domain knowledge to use the software and corresponding hardware in their own environment.


\section*{Timeline}
%!!! We don't actually have accurate dates for stuff in 2015

\begin{tabular}{ | l | l |}
  \hline
  Date			& Task \\ \hline
  Fri 15 August 	& \emph{Project proposal and project summary due to Coordinator} \\ \hline
  August		& Hardware shortlisting / testing \\  \hline
  25--29 August 	& \emph{Project proposal talk presented to research group} \\ \hline
  September		& Literature review \\ \hline
  Fri 19 September 	& \emph{Draft literature review due to supervisor(s)} \\ \hline
  October - November	& Core Hardware / Software development \\ \hline
  Fri 24 October 	& \emph{Literature Review and Revised Project Proposal due to Coordinator} \\ \hline
  November - February	& \emph{End of year break} \\ \hline
  February		& Write dissertation \\ \hline
  Thu 16 April 		& \emph{Draft dissertation due to supervisor} \\ \hline
  April - May		& Improve robustness and API \\ \hline
  Thu 30 April 		& \emph{Draft dissertation available for collection from supervisor} \\ \hline
  Fri 8 May 		& \emph{Seminar title and abstract due to Coordinator} \\ \hline
  Mon 25 May 		& \emph{Final dissertation due to Coordinator} \\ \hline
  25--29 May		& \emph{Seminar Presented to Seminar Marking Panel} \\ \hline
  Thu 28 May 		& \emph{Poster Due} \\ \hline
  Mon 22 June		& \emph{Corrected Dissertation Due to Coordinator} \\
  \hline
\end{tabular}

\section*{Software and Hardware Requirements}
% Outline what your specific requirements will be with regard
% to software and hardware, but note that any special requests
% might need to be approved by your supervisor and the Head of
% Department.

A large part of this research project is determining the specific hardware and software that best fit the accessibility criteria. Because of this, an exhaustive list of software and hardware requirements are not given in this proposal.

A budget of up to \$300 has been allocated by my supervisor for project purchases. Some technologies with promise that will be investigated include;

\begin{description}
  \item[Raspberry Pi Model B+] Small form-factor Linux computer \hfill \\
  Available from \url{http://arduino.cc/en/Guide/Introduction}; \$38
  
  \item[Arduino Uno] Small form-factor microcontroller \hfill \\
  Available from \url{http://arduino.cc/en/Main/arduinoBoardUno}; \$36
  
  \item[Panasonic Grid-EYE] Infrared Array Sensor \hfill \\
  Available from \url{http://www3.panasonic.biz/ac/e/control/sensor/infrared/grid-eye/index.jsp}; approx. \$33
  
  \item[Passive Infrared Sensor] \hfill \\
  Available from various places; \$10--\$20
  
\end{description}

\newpage


\bibliography{../references/primary}


\end{document}
