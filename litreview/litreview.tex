\documentclass[../thesis/thesis.tex]{subfiles}
\begin{document}
\chapter{Literature Review}
\label{chap:litreview}

The proportion of elderly and mobility-impaired people is predicted to grow dramatically over the next century, leaving a large proportion of the population unable to care for themselves, and consequently less people able care for these groups. \cite{chan2009smart} With this issue looming, investments are being made into a variety of technologies that can provide the support these groups need to live independent of human assistance. 

With recent advancements in low cost embedded computing, such as the Arduino and Raspberry Pi, the ability to provide a set of interconnected sensors, actuators and interfaces to enable a low-cost `smart home for the disabled' is becoming increasingly achievable.

Sensing techniques to determine occupancy, the detection of the presence and number of people in an area, are of particular use to the elderly and disabled. Detection can be used to inform various devices that change state depending on the user's location, including the better regulation energy hungry devices to help reduce financial burden. Household climate control, which in some regions of Australia accounts for up to 40\% of energy usage \cite{abs4602} is one particular area in which occupancy detection can reduce costs, as efficiency can be increased dramatically with annual energy savings of up to 25\% found in some cases. \cite{beltran2013thermosense}
 
While many of the above solutions achieve excellent accuracies, in many cases they suffer from problems of installation logistics, difficult assembly, assumptions on user's technology ownership and component cost. In a smart home for the disabled, accuracy is important, but accessibility is paramount.

The goal of this research project is to devise an occupancy detection system that forms part of a larger `smart home for the disabled' that meets the following accessibility criteria;

\begin{itemize}
 \item \emph{Low Cost}: The set of components required should aim to minimise cost, as these devices are intended to be deployed in situations where the serviced user may be financially restricted.
 
 \item \emph{Non-Invasive}: The sensors used in the system should gather as little information as necessary to achieve the detection goal; there are privacy concerns with the use of high-definition sensors.
 
 \item \emph{Energy Efficient}: The system may be placed in a location where there is no access to mains power (i.e. roof), and the retrofitting of appropriate power can be difficult; the ability to survive for long periods on only battery power is advantageous.
 
 \item \emph{Reliable}: The system should be able to operate without user intervention or frequent maintenance, and should be able to perform its occupancy detection goal with a high degree of accuracy.
\end{itemize}

To create a picture of what options there are in this sensing area, a literature review of the available sensor types and wireless sensor architectures is needed. From this list, proposed solutions will be compared against the aforementioned accessibility criteria to determine their suitability.

\section{Sensors}
\label{sec:litreview:sensors}

To achieve the accessibility criteria, a wide variety of sensing approaches must be considered. It can be difficult to approach the board variety of sensor types in the field, so a structure must be developed through which to evaluate them. \cite{teixeira2010survey} proposes a 5-element human-sensing taxonomy that is very useful in considering the broad requirements of sensors for this research.

The taxonomic elements are;
\begin{enumerate}
 \item \emph{Presence}: Is there anyone present in the sensed area?
 \item \emph{Count}: How many people are there in the sensed area?
 \item \emph{Location}: Where are the people in the sensed area?
 \item \emph{Track}: Where do the people move in the sensed area? (local identification)
 \item \emph{Identity}: Who are the people in the sensed area? (global identification)
\end{enumerate}

These elements are particularly useful when it comes to comparing the different qualities of sensors, and to determine which sensors are useful for our particular purpose. At a fundamental level, this research project requires a sensor system that provides both Presence and Count information. To assist with the reduction of privacy concerns, excluding systems that permit Identity will generally result in a less invasive system also. The presence of Location or Track are irrelevant to our project's goals, but overall, minimising these elements should in most cases help to maximise the energy efficiency of the system also.

\cite{teixeira2010survey} also proposes a measurable human trait taxonomy (see \fref{fig:litreview:taxonomy}), which we use in this literature review as a structure through which we describe different sensor types.

\begin{figure}
\centering
\begin{forest}
[Sensors
  [Intrinsic
    [Static]
    [Dynamic]
  ]
  [Extrinsic
    [Instrumented]
    [Correlative]
  ]
]
\end{forest}
\caption{Sensor taxonomy}
\label{fig:litreview:taxonomy}
\end{figure}

\subsection{Intrinsic traits}
\label{subsec:litreview:sensors:intrinsic}

The first set of sensors discussed are sensors detecting intrinsic traits. Intrinsic traits are those which can be sensed that are a direct property of being human. Intrinsic traits are particularly useful, as in many situations they are guaranteed to be present if a human is present. However, they do have varying degrees of detectability and differentiation between people. Two main subcategories of these sensor types are static and dynamic traits.

Static traits are physiologically derived, and are present with most (alive) humans.

One key static trait that can be used for human sensing is that of thermal emissions. All humans emit a large amount of thermal radiation in both resting and active states. The heat signatures of these emissions could potentially be measured with some apparatus, counted, and used to provide Presence and Count information to a sensor system, without providing Identity information. Once such paper that proposes this is \cite{beltran2013thermosense}, which uses a type of thermal sensor known as an infrared array sensor. This sensor is much like a camera, in that it has a field of view which is divided into an $8\times8$ grid of detected temperatures. This sensor is mounted on an embedded device on the roof, along with a \pir, and uses a variety of classification algorithms to detect human heat signatures within the raw thermal and motion data it collects. Such a system achieves Root Mean Squared Error $\approx0.35$ persons. There are few drawbacks with this approach.

Another static trait is that of \cdi emissions, which, like thermal emissions, are present in both resting and active states. By measuring the buildup of \cdi within a given area, one can use a variety of mathematical models of human \cdi production to determine how many people there are likely in a space to have that amount of \cdi produced. This was trialled as part of a sensor fusion in \cite{hailemariam2011real}, within the context of an office environment, achieving a $\approx94\%$ accuracy. Such a sensing system could provide both the Presence and Count information, and exclude the Identity information as required. However, a \cdi based detection mechanism has serious drawbacks, discussed in \cite{fisk2006accuracy}; the \cdi feedback mechanism is very slow in nature, taking hours of continuous occupancy to correctly identify the presence of people. In a home environment, people are more likely to be moving between rooms than an office, so the system may have a more difficult time detecting in that situation. Similarly, such systems can be interfered with by other elements that control the \cdi buildup in a space, like air conditioners, open windows, etc. This is also much more of a concern in a home environment compared to the studied office space, as the average home can have numerous such confounding factors.

The final static trait discussed is actual visual identification, achieved through the use of video or still-image cameras and advanced image processing algorithms. Video can be used in occupancy detection in several different ways, achieving different levels of accuracy and requiring different configurations. The first use of video, discussed in \cite{erickson2013poem} is the use of video as a ``optical turnstile''; the video system detects people and the direction they are moving in at each entrance and exit to an area, and uses that information to extrapolate the number of people within the turnstiled area; this system has up to a 94\% accuracy. However, the main issue with such a system applied to a home environment is the system assumes that there will be wide enough ``turnstile areas'', corridors of a fairly large area that connect different sections of a building, to use as detection zones. While such corridors exist in office environments, they are less likely to exist in home ones.

The second use of video is a ceiling-based video system \cite{serrano2013efficient}, it uses advanced image processing algorithms to count the number of people in the captured area. This system achieves a specificity of $\mathit{TP}/(\mathit{TP}+\mathit{FP})\approx97\%$ and a sensitivity $\mathit{TP}/(\mathit{TP}+\mathit{FN})\approx96\%$ (TP = true positives, FP = false positives, FN = false negatives). Such a system could be successfully applied to the home environment, as both it and the ``optical turnstile'' model provide Presence and Count information. However, these systems also allow Identity to be determined, and are thus very invasive. Such invasive systems cannot be used due to privacy concerns.


Dynamic traits are usually products of human activity, and thus can generally only be detected when a human is physically active or in motion.

The first discussed dynamic trait based system is \cite{hnat2012doorjamb}, an ultrasonic system that uses clusters of such sensors above doorframes to detect the height and direction of people travelling between rooms. This acts as a turnstile based system, much like \cite{erickson2013poem}, but augments this with an understanding of the model of the building to error correct for invalid and impossible movements brought about from sensing errors. This system provides an overall room-level tracking accuracy of 90\%, however to achieve this accuracy, individuals are intended to be tracked using their heights, which is a privacy consideration. The system can also suffer from problems with error propagation, as there are possibilities of ``phantom'' people entering a room due to sensing errors.

The second dynamic trait system is based upon a \pir connected to a transmitter of some description. A solely \pir based system is used within \cite{hailemariam2011real}, with the motion of the sensor being averaged over several different time intervals, and fed into a decision tree classifier. This \pir system alone produced a $\approx98\%$ accuracy. However, such a system, due to only motion detection capabilities, can only provide Presence information, and is unable to provide Count information.

\subsection{Extrinsic traits}
\label{subsec:litreview:sensors:extrinsic}
The second set of sensors discussed are sensors detecting extrinsic traits. Extrinsic traits are those which are actually other environmental changes that are caused by or correlated with human presence. These traits generally present a less accurate picture, or require the sensed humans to be in some way ``tagged'', but they are generally also easier to sense in of themselves.

The sensors in this category have been divided into two subcategories for our purposes that are not discussed in \cite{teixeira2010survey}. The first of these categories is instrumented approaches; these require that detectable humans carry with them some device that is detected as a proxy for the human themselves.

The most obvious of these approaches is a specially designed device. \cite{li2012measuring} uses RFID tags placed on building occupant's persons and a set of transmitters to triangulate the tags and place them within different thermal zones for the use of the HVAC system. For stationary occupants, there was a detection accuracy of $\approx88\%$, and for occupants who were mobile, the accuracy was $\approx62\%$. Such a system could be repurposed for the home, however, these systems raise issues in a home environment as it requires individuals to be constantly carrying their sensors, which is not the case in such an environment. Additionally, the accuracy for this system is not necessarily high enough for a home environment, where much smaller rooms are used.

To make extrinsic detection more reliable, \cite{kleiminger2013inferring} leverages a device people are likely to carry around; wifi enabled smart phones. They propose the \textit{homeset} algorithm, which uses the phones to scan the visible wifi networks, and from that information estimate if the users are at home or out and about by ``triangulating'' their position from the visible wifi networks. This solution does not provide the fine-grained Presence data that we need, as it is only able to triangulate the phone's position very roughly with the wireless network detection information.

\cite{balaji2013sentinel} also leverages smart phones to determine occupancy, but in a more broad enterprise environment: Wireless association logs are analysed to determine which access points in a building a given user is connected to. If this access point falls within the radio range of their designated ``personal space'', they are considered to be occupying that personal space. This technique cannot be applied to a home environment, as there are usually not multiple wireless hotspots.

Finally, \cite{gupta2009adding} uses specifically the GPS functions of the smartphone to perform optimisation on heating and cooling systems by calculating the ``travel-to-home'' time of occupants at all times and ensuring at every distance the house is minimally heated such that if the user were to travel home, the house would be at the correct temperature when they arrived. While this system does achieve similar potential air-conditioning energy savings, it is not room-level modular, and also presupposes a user whose primary energy costs are from incorrect heating when away from home, which isn't necessarily the case for this demographic.

The second of these subcategories are correlative approaches. These approaches analyse data that is correlated with human activity, but does not require a specific device to be present on each human that is tracked with the system.

The first of these approaches is \cite{kleiminger2013occupancy}, which attempts to measure electricity consumption and use such data to determine Presence. Electricity data was measured at two different levels of granularity; the whole house level with a smart meter, and the consumption of specific appliances through smart plugs. This data was then processed by a variety of classifiers to achieve a classification accuracy of more than 80\%. Such a system presents a low-cost solution to occupancy, however it is not sufficiently granular in either the detection of multiple occupants, or the detection of occupants in a specific room.

% The second of these approaches is \cite{ting2013occupancy}, which attempts to measure network traffic, among other things, to determine Presence. 
% TODO: Unable to find paper corresponding to this Poster Abstract?

\subsection{Analysis}
\label{subsec:litreview:sensors:analysis}

\begin{table}
\begin{threeparttable}
\begin{tabularx}{\textwidth}{|l|l|l||l||l|l|}
\cline{2-6}
\multicolumn{1}{r|}{}		    	& \multicolumn{2}{c||}{Requires} & Excludes & \multicolumn{2}{c|}{Irrelevant} \\
\cline{2-6}
\multicolumn{1}{r|}{}		    	& \csbox{Presence} & \csbox{Count} & \csbox{Identity} & \csbox{Location} & \csbox{Track} \\
\cline{1-6}

\underline{Intrinsic} 			& & & & & \\
\hspace{3mm}\textit{Static} 		& & & & & \\
\hspace{8mm}Thermal 			& \cmark & \cmark & \cmark & \cmark &  \\
\hspace{8mm}\cdi			& \cmark & \cmark & \cmark &  &  \\
\hspace{8mm}Video			& \cmark & \cmark & \xmark & \cmark & \cmark \\

\hspace{3mm}\textit{Dynamic} 		& & & & & \\
\hspace{8mm}Ultrasonic	 		& \cmark & \cmark & \xmark & & \cmark \\
\hspace{8mm}PIR		 		& \cmark & \xmark & \cmark &  &  \\

					& & & & & \\

\underline{Extrinsic}			& & & & & \\
\hspace{3mm}\textit{Instrumented} 	& & & & & \\
\hspace{8mm}RFID 			& \cmark\ssup & \cmark & \cmark & \cmark & \\
\hspace{8mm}WiFi assoc.\tsup		& \cmark\ssup & \cmark & \xmark & \cmark & \\
\hspace{8mm}WiFi triang.\tsup		& \cmark\ssup & \cmark & \xmark & & \\
\hspace{8mm}GPS\tsup			& \cmark\ssup & \xmark & \cmark & \cmark & \\

\hspace{3mm}\textit{Correlative} 	& & & & & \\
\hspace{8mm}Electricity 		& \cmark\ssup & \xmark & \cmark & & \\
%\hspace{8mm}Network			& \cmark\ssup & \xmark & \cmark & & \\ TODO: Uncomment once above todo is sorted

\cline{1-6}
\end{tabularx}
\begin{tablenotes}
\item \ssup  Doesn't provide data at required level of accuracy for home use.
\item \tsup  Uses smartphone as detector.
\end{tablenotes}
\end{threeparttable}
\caption{Comparison of different sensors and project requirements}
\label{tab:litreview:taxonomycomp}
\end{table}

From these various sensor options, there are a few candidates that provide the necessary initial sensing elements (Presence and Count); these are thermal, \cdi, Video, Ultrasonic, RFID and WiFi association and triangulation based methods. All sensing options are compared on \Fref{tab:litreview:taxonomycomp}.

In the context of our four accessibility criteria, \cdi sensing, has several reliability drawbacks, the predominant ones being a large lag time to receive accurate occupancy information and interference from a variety of air conditioning sources which can modify the \cdi concentration in the room in unexpected ways.

Video-based sensing methods suffer from invasiveness concerns, as they by design must have a constant video feed of all detected areas.

Ultrasonic methods have particular issues with reliability when a user falls outside the prescribed height bounds of normal humans. Wheelchair bound occupants, a core demographic of this sensing system, would be incompatible with such an approach. Their wheelchair may also interfere with height measurement results. Ultrasonic methods also provide weak Identity information through height detection.

RFID sensing, also has several drawbacks, predominantly being that it is difficult value proposition to get house occupants to carry RFID tags with them continuously. Another drawback is that the triangulation methods discussed are too unreliable to place occupants in specific rooms in many cases.

WiFi association is not granular enough for home use, as the original enterprise use case presupposed a much larger area, as well as multiple wireless access points, neither of which a typical home environment have.

WiFi triangulation is a good candidate for the home, as there are most likely neighbouring wireless networks that can be used as virtual landmarks. However, it suffers from the same granularity problems as WiFi association, as these signals are not specific enough to pinpoint a person to a specific room.

For smartphone based approaches generally, it is also more difficult in home environments to ensure that occupants are carrying their smartphones with them at all times.  Another issue with smart phones is that they represent an expense that the target markets of the elderly and the disabled may not be able to afford.

Finally, we have thermal sensing. It provides both Presence and Count information, as it uses occupants' thermal signatures to determine the presence of people in a room. It does not however provide Identity information, as thermal signatures are not sufficiently unique with the technologies used to distinguished between occupants. Such a sensor system is presented as low-cost and energy efficient within \cite{beltran2013thermosense}, is non-invasive by design and can reliably detect occupants with a very low root mean squared error.

\section{Thermal sensors}
\label{sec:litreview:thermalsensors}
Above we concluded that thermal sensors are the best candidates for this project. In this section we discuss the thermal sensing field in more detail.

A primary static/dynamic sensor fusion system in this field is the Thermosense system, \cite{beltran2013thermosense} a \pir and \iar\footnote{Phillips GridEYE; approx \$30} used to subdivide an area into an $8\times8$ grid of sections from which temperatures can be derived. This sensor system is attached to the roof on a small embedded controller which is responsible for collecting the data and transmitting it back to a larger computer via low powered wireless protocols.

The Thermosense system develops a thermal background map of the room using an \emwa over a 15 minute time window (if no motion is detected). If the room remains occupied for a long period, a more complex scaling algorithm is used which considers the coldest points in the room empty, and averages them against the new background, then performs \emwa with a lower weighting.

This background map is used as a baseline to calculate standard deviations of each grid area, which are then used to determine several characteristics to be used as feature vectors for a variety of classification approaches. The determination of the feature vectors was subject to experimentation, with the differences at each grid element too susceptible to individual room conditions to be used as feature vectors. Instead, a set of three different features was designed; the number of temperature anomalies in the space, the number of groups of temperature anomalies, and the size of the largest anomaly in the space. These feature vectors were compared against three classification approaches; K-Nearest Neighbors, Linear Regression and an a feed-forward Artificial Neural Network of one hidden later and 5 perceptions. All three classifiers achieved a Root Mean Squared Error (RMSE) within $0.38\pm0.04$. This final classification is subject to a final averaging process over a 4 minute window to remove the presence of independent errors from the raw classification data.

\section{Architecture}
\label{sec:litreview:architecture}
Beyond specific sensor design, a large emphasis of this project is to create a system that is correctly architected to operate in a real-world \iot environment, as the key advantage of \iot ``things'' is in the ``disruptive level of innovation''\cite{atzori2010internet} brought about by their ability to be combined in ways unforeseen (yet still enabled) by their creators. This architecture involves careful consideration of the embedded hardware that will drive the system, as well as the communications protocols utilised between the sensor and devices interested in the sensor's information.

\subsection{Protocols}
\label{subsec:litreview:architecture:protocols}
% From; https://openwsn.atlassian.net/wiki/pages/viewpage.action?pageId=29196353
\begin{table}
\centering
\begin{tabular}{|c|c|}
\hline
\textbf{Application} & \acs{coap} \\ \hline
\textbf{Transport} & UDP \\ \hline
\textbf{IP / Routing} & IETF \acs{roll} \\ \hline
\textbf{Adaptation} & IETF \acs{lowpan} \\ \hline
\textbf{Medium Access} & IEEE \lmed \\ \hline
\textbf{Physical} & IEEE \lphy \\ \hline
\end{tabular}
\caption{Proposed protocol stack}
\label{tab:litreview:protostack}
\end{table}

As the system will most likely be battery powered, it is important to prioritise those protocols and architectures that minimise power usage. The system will also ideally exist in a system with other identical sensors (one for each room in a house), thus it is important to prioritise those protocols which allow multiple identical systems to coexist on the same network without conflict, and to be uniquely addressable and identifiable. In recent years, many developments have been made in the \iot arena, with standards emerging specifically designed for low-power embedded devices to communicate between themselves and bigger systems that address these and other unique needs, across the entire protocol stack. 

In addition to this protocol stack, at the application layer, \cite{guinard2012search} proposes the use of \rest over \ws  as a method of exchanging information between sensor systems. Their data suggests that \rest is easier to use than \ws, and the key advantage of a \ws based approach is its ability to represent much more complex data and abstractions, which are unnecessary in this project's situation.

\cite{palattella2013standardized} proposes a protocol stack that aligns with the above requirements, the key advantage of which is being that an implementation of that stack is completely standardised, based on TCP/IP, uses the latest IEEE and IETF \iot standards, and is free from proprietary protocol restrictions (unlike ZigBee 1.0 devices, for instance). \Fref{tab:litreview:protostack} shows the full stack proposed. The key components of this proposal are the introduction of \acs{coap} at the application layer, \acs{roll} at the IP / Routing layer and \acs{lowpan} at the Adaptation layer.

\coap \cite{kovatsch2013coap} is an application layer protocol designed to replace HTTP as a way of transmitting RESTful information between clients. The chief advantage of \coap over HTTP is it compresses the broad-strokes of the HTTP feature set into a binary language that is much more suitable for transmission over low-bandwidth and low-power links, such as those discussed here.

\roll \cite{rfc6550} is a routing protocol designed for low power environments, allowing low power nodes to create and maintain a mesh network between themselves, allowing, among other things, the routing of packets to a ``root'' node and back again. \roll is particularly suited to the routing situation of our proposed architecture, as individual sensors do not need to communicate with one another, but rather report back to a larger node (further discussed in \fref{subsec:litreview:architecture:devices}).

\lowpan \cite{shelby20116lowpan} is a compression and formatting specification to allow IPv6 packets to be sent over an \lwifi based network. Optimisations are found in the reduction of the size of \lowpan packets, IPv6 addresses as well as redesigning core Internet Protocol algorithms so that they can run with low power consumption on participating devices.

% TODO: Discuss ZigBee, and problems with GPL, and proprietary protocol

\subsection{Devices}
\label{subsec:litreview:architecture:devices}
\begin{figure}
\centering
\begin{tikzpicture}[node distance=1.7cm]
\node (interwebs) [cbox] {Internet};
\node (http) [dashbox, left=of interwebs] {\small HTTP};
\node (rpi) [box, below=of http] {Smart Gateway / Processor};
\node (coap) [dashbox, right=of rpi] {\small \acs{coap}};
\node (mesh) [cbox, right=of coap] {WPAN};
\node (embed1) [box, below left=of mesh] {Sensor};
\node (embed2) [box, below right=of mesh] {Sensor};

\draw [line] (interwebs) -- (http);
\draw [line] (http) -- (rpi);
\draw [line] (rpi) -- (coap);
\draw [line] (coap) -- (mesh);
\draw [line] (mesh) -- (embed1);
\draw [line] (mesh) -- (embed2);
\end{tikzpicture}
\caption{Proposed system architecture}
\label{fig:litreview:devices}
\end{figure}

In addition to the protocol stack used, how these nodes relate to each other is also an important consideration. Part of what will inform these decisions are the requisite processing power and internet connectivity required to successfully execute all elements of this project. \cite{kovatsch2013coap} describes three classes of resource constrained devices that would benefit from \coap, and each can provide different levels of security for an IP stack;

\begin{itemize}
 \item \textbf{Class 0}: ``not capable of running an RFC-compliant IP stack in a secure manner. They require application-level gateways to connect to the Internet.''
 \item \textbf{Class 1}: Able to connect to the internet with some ``integrated security mechanisms''. Are unable to employ full HTTP with TLS.
 \item \textbf{Class 2}: Normal Internet nodes, able to use the full HTTP stack with TLS.
\end{itemize}

The devices that we propose the sensors will connect to are the likes of the Arduino, which can be generally classified as class 0 or possibly class 1 devices. Because of this, \cite{guinard2011internet} proposes the use of a ``Smart Gateway'' system to bridge the wider internet and these sensor systems. This gateway would be able to communicate with the sensor systems over \coap and \lwifi, as well as receive API requests via HTTP from a traditional TCP/IP network to forward on to these sensors.

In addition to this, the Thermosense paper \cite{beltran2013thermosense} proposes several different algorithms to process the raw sensing data into the occupancy estimates (further discussed in \fref{sec:litreview:thermalsensors}), all of which are fairly computationally expensive. Because of this, it would be non-trivial to implement these algorithms on the embedded sensing devices themselves. This problem is already resolved in our proposed system, as the aforementioned ``Smart Gateway'' can easily also take on the task of processing the raw sensor data into estimates which it can relay to interested parties over its HTTP-based API. A visualisation of this proposed system can be seen in \fref{fig:litreview:devices}.

\section{Research Gap}
\label{sec:litreview:researchgap}
Throughout this discussion of the area of wireless occupancy sensors within the \iot it can be seen that there is a clear research gap within the area of occupancy. No group could be found who has assembled an occupancy sensor that optimises these ares of Low Cost, Non-Invasiveness, Energy Efficiency and Reliability into a architected software and hardware package that can be integrated like any other Thing into the \iot.

This is a key research area, because, as we have previously discussed, the true ``disruptive level of innovation''\cite{atzori2010internet} the \iot provides can only be realised once a novel idea has been properly packaged as a Thing, rather than as a research curiosity. Packaging something as a Thing requires careful consideration of the best sensing systems, the best hardware to run those systems on, the best protocols to allow these Things to communicate, and the best device architecture to enable that communication. The state of the art in all these areas have been discussed throughout this literature review.

A key part of enabling the ``smart home for the disabled'' is creating a set of Things that can improve quality of life for those people. Our proposed Thing has clearly demonstrated this potential.


\ifcsdef{mainfile}{}{\bibliography{../references/primary}}
\end{document}