\documentclass[../thesis/thesis.tex]{subfiles}
\begin{document}
\chapter{Literature Review}

The proportion of elderly and mobility-impaired people is predicted to grow dramatically over the next century, leaving a large proportion of the population unable to care for themselves, and consequently less people able care for these groups. \cite{chan2009smart} With this issue looming, investments are being made into a variety of technologies that can provide the support these groups need to live independent of human assistance. 

With recent advancements in low cost embedded computing, such as the Arduino \cite{Ardunio} and Raspberry Pi, \cite{RPi} the ability to provide a set of interconnected sensors, actuators and interfaces to enable a low-cost `smart home for the disabled' is becoming increasingly achievable.

Sensing techniques to determine occupancy, the detection of the presence and number of people in an area, are of particular use to the elderly and disabled. Detection can be used to inform various devices that change state depending on the user's location, including the better regulation energy hungry devices to help reduce financial burden. Household climate control, which in some regions of Australia accounts for up to 40\% of energy usage \cite{abs4602} is one particular area in which occupancy detection can reduce costs, as efficiency can be increased dramatically with annual energy savings of up to 25\% found in some cases. \cite{beltran2013thermosense}
 
While many of the above solutions achieve excellent accuracies, in many cases they suffer from problems of installation logistics, difficult assembly, assumptions on user's technology ownership and component cost. In a smart home for the disabled, accuracy is important, but accessibility is paramount.

The goal of this research project is to devise an occupancy detection system that forms part of a larger `smart home for the disabled' that meets the following accessibility criteria;

\begin{itemize}
 \item \emph{Low Cost}: The set of components required should aim to minimise cost, as these devices are intended to be deployed in situations where the serviced user may be financially restricted.
 
 \item \emph{Non-Invasive}: The sensors used in the system should gather as little information as necessary to achieve the detection goal; there are privacy concerns with the use of high-definition sensors.
 
 \item \emph{Energy Efficient}: The system may be placed in a location where there is no access to mains power (i.e. roof), and the retrofitting of appropriate power can be difficult; the ability to survive for long periods on only battery power is advantageous.
 
 \item \emph{Reliable}: The system should be able to operate without user intervention or frequent maintenance, and should be able to perform its occupancy detection goal with a high degree of accuracy.
\end{itemize}

To create a picture of what options there are in this sensing area, a literature review of the available sensor types and wireless sensor architectures is needed. From this list, proposed solutions will be compared against the aforementioned accessibility criteria to determine their suitability.

\section{Sensors}

To achieve the accessibility criteria, a wide variety of sensing approaches must be considered. It can be difficult to approach the board variety of sensor types in the field, so a structure must be developed through which to evaluate them. \cite{teixeira2010survey} proposes a 5-element human-sensing taxonomy that is very useful in considering the broad requirements of sensors for this research.

The taxonomic elements are;
\begin{enumerate}
 \item \emph{Presence}: Is there anyone present in the sensed area?
 \item \emph{Count}: How many people are there in the sensed area?
 \item \emph{Location}: Where are the people in the sensed area?
 \item \emph{Track}: Where do the people move in the sensed area? (local identification)
 \item \emph{Identity}: Who are the people in the sensed area? (global identification)
\end{enumerate}

These elements are particularly useful when it comes to comparing the different qualities of sensors, and to determine which sensors are useful for our particular purpose. At a fundamental level, this research project requires a sensor system that provides both Presence and Count information. To assist with the reduction of privacy concerns, excluding systems that permit Identity will generally result in a less invasive system also. The presence of Location or Track are irrelevant to our project's goals, but overall, minimising these elements should in most cases help to maximise the energy efficiency of the system also.

\cite{teixeira2010survey} also proposes a measurable human trait taxonomy, which we use in this literature review as a structure through which we describe different sensor types.

The first set of sensors discussed are sensors detecting intrinsic traits. Intrinsic traits are those which can be sensed that are a direct property of being human. Intrinsic traits are particularly useful, as in many situations they are guaranteed to be present if a human is present. However, they do have varying degrees of detectability and differentiation between people. Two main subcategories of these sensor types are static and dynamic traits.

Static traits are physiologically derived, and are present with most (alive) humans.

% Thermosense

% Another sensor system in this area is that of CO\textsubscript{2}, which is discussed in \cite{fisk2006accuracy}

Dynamic traits are usually products of human activity, and thus can generally only be detected when a human is physically active or in motion.


\begin{table}
\begin{tabularx}{\textwidth}{|l|c|c||c||c|c|}
\cline{2-6}
\multicolumn{1}{r|}{}		    & \multicolumn{2}{c||}{Requires} & Excludes & \multicolumn{2}{c|}{Irrelevant} \\
\cline{2-6}
\multicolumn{1}{r|}{}		    & \csbox{Presence} & \csbox{Count} & \csbox{Identity} & \csbox{Location} & \csbox{Track} \\
\cline{1-6}

\underline{Intrinsic} 			& & & & & \\
\hspace{3mm}\textit{Static} 		& & & & & \\
\hspace{6mm}Thermal 			& \cmark & \cmark & \cmark & \cmark &  \\
\hspace{6mm}CO\textsubscript{2}		& \cmark & \cmark & \cmark &  &  \\
\hspace{6mm}Video			& \cmark & \cmark & \xmark & \cmark & \cmark \\

\hspace{3mm}\textit{Dynamic} 		& & & & & \\
\hspace{6mm}Ultrasonic	 		& \cmark & \cmark & \xmark & & \cmark \\
\hspace{6mm}PIR		 		& \cmark & \xmark & \cmark &  &  \\

					& & & & & \\

\underline{Extrinsic}			& & & & & \\
\hspace{3mm}\textit{Instrumented} 	& & & & & \\
\hspace{6mm}RFID 			& \cmark & \cmark & \cmark & \cmark & \\
\hspace{6mm}Smart phone			& \cmark & \cmark & \xmark & \cmark &  \\
\hspace{6mm}GPS 			& \cmark & \xmark & \cmark & \cmark & \\

\hspace{3mm}\textit{Correlative} 	& & & & & \\
\hspace{6mm}Electricity 		& \cmark & \xmark & \cmark & & \\
\hspace{6mm}Network			& \cmark & \xmark & \cmark & & \\

\cline{1-6}
\end{tabularx}
\caption{Comparison of different sensors and project requirements}
\end{table}

\section{Thermal sensors}

A primary static/dynamic sensor fusion system in this field is the Thermosense system, \cite{beltran2013thermosense} a PIR\footnote{Passive Infrared Array} and thermal grid array\footnote{Phillips GridEYE; approx \$30} used to subdivide an area into an 8x8 grid of sections from which temperatures can be derived. This sensor system is attached to the roof on a small embedded controller which is responsible for collecting the data and transmitting it back to a larger computer via low powered wireless protocols.

The Thermosense system develops a thermal background map of the room using a an exponential weighted moving average (EMWA) over a 15 minute time window (if no motion is detected). If the room remains occupied for a long period, a more complex scaling algorithm is used which considers the coldest points in the room empty, and averages them against the new background, then performs EMWA with a lower weighting.

This background map is used as a baseline to calculate standard deviations of each grid area, which are then used to determine several characteristics to be used as feature vectors for a variety of classification approaches. The determination of the feature vectors was subject to experimentation, with the differences at each grid element too susceptible to individual room conditions to be used as feature vectors. Instead, a set of three different features was designed; the number of temperature anomalies in the space, the number of groups of temperature anomalies, and the size of the largest anomaly in the space. These feature vectors were compared against three classification approaches; K-Nearest Neighbors, Linear Regression and an a feed-forward Artificial Neural Network of one hidden later and 5 perceptions. All three classifiers achieved a Root Mean Squared Error (RMSE) within $0.38\pm0.04$. This final classification is subject to a final averaging process over a 4 minute window to remove the presence of independent errors from the raw classification data.

\section{Architecture}



\ifcsdef{mainfile}{}{\bibliography{../references/primary}}
\end{document}