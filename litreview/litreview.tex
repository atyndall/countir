\documentclass[../thesis/thesis.tex]{subfiles}
\begin{document}
\ifcsdef{mainfile}{}{%
  \renewcommand{\thetitle}{Developing a robust system for occupancy detection in the household \linebreak\linebreak Literature Review}%
  \maketitle%
  \tableofcontents%
}
\chapter{Literature Review}
\label{chap:litreview}
\label{sec:litreview:sensors}

To achieve the accessibility criteria, a wide variety of sensing approaches must be considered. It can be difficult to approach the board variety of sensor types in the field, so a structure must be developed through which to evaluate them. Teixeira, Dublon and Savvides \cite{teixeira2010survey} propose a 5-element human-sensing criteria which provides a structure through which we may define the broad quantitative requirements of different sensors.

These quantitative requirements can be used to exclude sensing options that clearly cannot meet the requirements before the more specific qualitative accessibility criteria will be considered for those remaining sensors. 

The quantitative criteria elements are;
\begin{enumerate}
 \item \emph{Presence}: Is there any occupant present in the sensed area?
 \item \emph{Count}: How many occupants are there in the sensed area?
 \item \emph{Location}: Where are the occupants in the sensed area?
 \item \emph{Track}: Where do the occupants move in the sensed area? (local identification)
 \item \emph{Identity}: Who are the occupants in the sensed area? (global identification)
\end{enumerate}

At a fundamental level, this research project requires a sensor system that provides both Presence and Count information. To assist with the reduction of privacy concerns, excluding systems that permit Identity will generally result in a less invasive system also. The presence of Location or Track are irrelevant to our project's goals, but overall, minimising these elements should in most cases help to maximise the energy efficiency of the system also.

Teixeira, Dublon and Savvides \cite{teixeira2010survey} also propose a measurable occupancy sensor taxonomy (see \Fref{fig:litreview:taxonomy}), which categorises different sensing systems in terms of what information they use as a proxy for human-sensing. We use this taxonomy here as a structure through which we group and discuss different sensor types.

\begin{figure}
\centering
\begin{forest}
[Sensors
  [Intrinsic
    [Static]
    [Dynamic]
  ]
  [Extrinsic
    [Instrumented]
    [Correlative]
  ]
]
\end{forest}
\caption{Taxonomy of occupancy sensors}
\label{fig:litreview:taxonomy}
\end{figure}

\section{Intrinsic traits}
\label{subsec:litreview:sensors:intrinsic}

Intrinsic traits are those which can be sensed that are a direct property of being a human occupant. Intrinsic traits are particularly useful, as in many situations they are guaranteed to be present if an occupant is present. However, they do have varying degrees of detectability and differentiation between occupants. Two main subcategories of these sensor types are static and dynamic traits.

\subsection{Static traits}
\label{subsubsec:litreview:sensors:intrinsic:static}
Static traits are physiologically derived, and are present with most (living) occupants. One key static trait that can be used for occupant sensing is that of thermal emissions. All human occupants emit distinctive thermal radiation in both resting and active states. The heat signatures of these emissions could potentially be measured with some apparatus, counted, and used to provide Presence and Count information to a sensor system, without providing Identity information.

Beltran, Erickson and Cerpa \cite{beltran2013thermosense} propose Thermosense, a system that uses a type of thermal sensor known as an \iar. This sensor is much like a camera, in that it has a field of view which is divided into ``pixels''; in this case an $8\times8$ grid of detected temperatures. This sensor is mounted on an embedded device on the ceiling, along with a \pir, and uses a variety of classification algorithms to detect human heat signatures within the raw thermal and motion data it collects. Thermosense achieves Root Mean Squared Error $\approx0.35$ persons, meaning the standard deviation between Thermosense's occupancy predictions and the actual occupancy number was $\approx0.35$.

Another static trait is that of \cdi emissions, which, like thermal emissions, are emitted by human occupants in both resting and active states. By measuring the buildup of \cdi within a given area, one can use a variety of mathematical models of human \cdi production to determine the likely number of occupants present. Hailemariam \etal \cite{hailemariam2011real} trialled this as part of a sensor fusion within the context of an office environment, achieving a $\approx94\%$ accuracy. Such a sensing system could provide both the Presence and Count information, and exclude the Identity information as required. However, a \cdi based detection mechanism has serious drawbacks, discussed by Fisk, Faulkner and Sullivan \cite{fisk2006accuracy}: The \cdi feedback mechanism is very slow, taking hours of continuous occupancy to correctly identify the presence of people. In a residential environment, occupants are more likely to be moving between rooms than an office, so the system may have a more difficult time detecting in that situation. Similarly, such systems can be interfered with by other elements that control the \cdi buildup in a space, like air conditioners, open windows, etc. This is also much more of a concern in a residential environment compared to the studied office space, as the average residence can have numerous such confounding factors that cannot easily be controlled for.

Visual identification can be, achieved through the use of video or still-image cameras and advanced image processing algorithms. Video can be used in occupancy detection in several different ways, achieving different levels of accuracy and requiring different configurations. The first use of video, POEM, proposed by Erickson, Achleitner and Cerpa \cite{erickson2013poem} is the use of video as a ``optical turnstile''; the video system detects potential occupants and the direction they are moving in at each entrance and exit to an area, and uses that information to extrapolate the number of occupants within the turnstiled area; this system has up to a 94\% accuracy. However, the main issue with such a system applied to a residential environment is the system assumes that there will be wide enough ``turnstile areas'', corridors of a fairly large area that connect different sections of a building, to use as detection zones. While such corridors exist in office environments, they are less likely to exist in residential ones.

Another video sensor system is proposed by Serrano-Cuerda \etal \cite{serrano2013efficient}, that uses ceiling-based cameras and advanced image processing algorithms to count the number of people in the captured area. This system achieves a specificity of $\mathit{TP}/(\mathit{TP}+\mathit{FP})\approx97\%$ and a sensitivity $\mathit{TP}/(\mathit{TP}+\mathit{FN})\approx96\%$ (TP = true positives, FP = false positives, FN = false negatives). Such a system could be successfully applied to the residential environment, as both it and the ``optical turnstile'' model provide Presence and Count information. However, these systems also allow Identity to be determined, and thus are perceived as privacy-invasive. This perception leads to adoption and acceptance issues, which work against the ideal system's goals.

\subsection{Dynamic traits}
\label{subsubsec:litreview:sensors:intrinsic:dynamic}
Dynamic traits are usually products of human occupant activity, and thus can generally only be detected when a human occupant is physically active or in motion.

Ultrasonic systems, such as Doorjamb proposed by Hnat \etal \cite{hnat2012doorjamb}, use clusters of such sensors above doorframes to detect the height and direction of potential occupants travelling between rooms. This acts as a turnstile based system, much like POEM \cite{erickson2013poem}, but augments this with an understanding of the model of the building to error correct for invalid and impossible movements brought about from sensing errors. This system provides an overall room-level tracking accuracy of 90\%, however to achieve this accuracy, potential occupants are intended to be tracked using their heights, which has privacy implications. The system can also suffer from problems with error propagation, as there are possibilities of ``phantom'' occupants entering a room due to sensing errors.

Solely \pir based systems, like those used by Hailemariam \etal \cite{hailemariam2011real}, involve the motion of the sensor being averaged over several different time intervals, and fed into a decision tree classifier. This \pir system alone produced a $\approx98\%$ accuracy. However, such a system, due to only motion detection capabilities, can only provide Presence information, and is unable to provide Count information, nor detect motionless occupants.

\section{Extrinsic traits}
\label{subsec:litreview:sensors:extrinsic}
Extrinsic traits are those which are actually other environmental changes that are caused by or correlated with human occupant presence. These traits generally present a less accurate picture, or require the sensed occupants to be in some way ``tagged'', but they are generally also easier to sense in of themselves. The sensors in this category have been divided into two subcategories.

\subsection{Instrumented traits}
\label{subsubsec:litreview:sensors:extrinsic:instrumented}
One extrinsic trait category is instrumented approaches; these require that detectable occupants carry with them some device that is detected as a proxy for the occupant themselves.

The most obvious of these approaches is a specially designed device. Li \etal \cite{li2012measuring} use RFID tags placed on building occupant's persons and a set of transmitters to triangulate the tags and place them within different thermal zones for the use of the HVAC system. For stationary occupants, there was a detection accuracy of $\approx88\%$, and for occupants who were mobile, the accuracy was $\approx62\%$. Such a system could be re-purposed for the residence, however, these systems raise issues in a residential environment as it requires occupants to be constantly carrying their sensors, which is less likely in such an environment. Additionally, the accuracy for this system is not necessarily high enough for a residential environment, where much smaller rooms are used.

To make extrinsic detection more reliable, Li, Calis and Becerik-Gerber \cite{kleiminger2013inferring} leverage a common consumer device; wifi enabled smart phones. They propose the \textit{homeset} algorithm, which uses the phones to scan the visible wifi networks, and from that information estimate if the occupants are at home or out and about by ``triangulating'' their position from the visible wifi networks. This solution does not provide the fine-grained Presence data that we need, as it is only able to triangulate the phone's position very roughly with the wireless network detection information.

Balaji \etal \cite{balaji2013sentinel} also leverage smart phones to determine occupancy, but in a more broad enterprise environment: Wireless association logs are analysed to determine which access points in a building a given occupant is connected to. If this access point falls within the radio range of their designated ``personal space'', they are considered to be occupying that personal space. This technique cannot be applied to a residential environment, as there are usually not multiple wireless hotspots.

Finally, Gupta, Intille and Larson \cite{gupta2009adding} use specifically the GPS functions of the smartphone to perform optimisation on heating and cooling systems by calculating the ``travel-to-home'' time of occupants at all times and ensuring at every distance the house is minimally heated such that if the potential occupant were to travel home, the house would be at the correct temperature when they arrived. While this system does achieve similar potential air-conditioning energy savings, it is not room-level modular, and also presupposes an occupant whose primary energy costs are from incorrect heating when away from home, which isn't necessarily the case for this demographic.

\subsection{Correlative traits}
\label{subsubsec:litreview:sensors:extrinsic:correlative}
The second of these subcategories are correlative approaches. These approaches analyse data that is correlated with human occupant activity, but does not require a specific device to be present on each occupant that is tracked with the system.

The primary approach in this area is work done by Kleiminger \etal \cite{kleiminger2013occupancy}, which attempts to measure electricity consumption and use such data to determine Presence. Electricity data was measured at two different levels of granularity; the whole house level with a smart meter, and the consumption of specific appliances through smart plugs. This data was then processed by a variety of classifiers to achieve a classification accuracy of more than 80\%. Such a system presents a low-cost solution to occupancy, however it is not sufficiently granular in either the detection of multiple occupants, or the detection of occupants in a specific room.

\section{Analysis}
\label{subsec:litreview:sensors:analysis}

\begin{table}
\begin{threeparttable}
\begin{tabularx}{\textwidth}{|l|l|l||l||l|l|}
\cline{2-6}
\multicolumn{1}{r|}{}		    	& \multicolumn{2}{c||}{Requires} & Excludes & \multicolumn{2}{c|}{Irrelevant} \\
\cline{2-6}
\multicolumn{1}{r|}{}		    	& \csbox{Presence} & \csbox{Count} & \csbox{Identity} & \csbox{Location} & \csbox{Track} \\
\cline{1-6}

\underline{Intrinsic} 			& & & & & \\
\hspace{3mm}\textit{Static} 		& & & & & \\
\hspace{8mm}Thermal 			& \cmark & \cmark & \cmark & \cmark &  \\
\hspace{8mm}\cdi			& \cmark & \cmark & \cmark &  &  \\
\hspace{8mm}Video			& \cmark & \cmark & \xmark & \cmark & \cmark \\

\hspace{3mm}\textit{Dynamic} 		& & & & & \\
\hspace{8mm}Ultrasonic	 		& \cmark & \cmark & \xmark & & \cmark \\
\hspace{8mm}PIR		 		& \cmark & \xmark & \cmark &  &  \\

					& & & & & \\

\underline{Extrinsic}			& & & & & \\
\hspace{3mm}\textit{Instrumented} 	& & & & & \\
\hspace{8mm}RFID 			& \cmark\ssup & \cmark & \cmark & \cmark & \\
\hspace{8mm}WiFi assoc.\tsup		& \cmark\ssup & \cmark & \xmark & \cmark & \\
\hspace{8mm}WiFi triang.\tsup		& \cmark\ssup & \cmark & \xmark & & \\
\hspace{8mm}GPS\tsup			& \cmark\ssup & \xmark & \cmark & \cmark & \\

\hspace{3mm}\textit{Correlative} 	& & & & & \\
\hspace{8mm}Electricity 		& \cmark\ssup & \xmark & \cmark & & \\

\cline{1-6}
\end{tabularx}
\begin{tablenotes}
\item \ssup  Doesn't provide data at required level of accuracy for home use.
\item \tsup  Uses smartphone as detector.
\end{tablenotes}
\end{threeparttable}
\caption{Comparison of different sensors and project requirements}
\label{tab:litreview:taxonomycomp}
\end{table}

From these various sensor options, there are a few candidates that provide the necessary quantitative criteria (Presence and Count); these are thermal, \cdi, Video, Ultrasonic, RFID and WiFi association and triangulation based methods. All sensing options are compared on \Fref{tab:litreview:taxonomycomp}.

In the context of our four qualitative accessibility criteria, \cdi sensing has several reliability drawbacks, the predominant ones being a large lag time to receive accurate occupancy information and interference from a variety of air conditioning sources which can modify the \cdi concentration in the room in unexpected ways.

Video-based sensing methods suffer from invasiveness concerns, as they by design must have a constant video feed of all detected areas.

Ultrasonic methods suffer from reliability concerns when a user falls outside the prescribed height bounds of normal humans. Wheelchair bound occupants, a core demographic of our proposed sensing system, are not discussed in the Doorjamb paper. Their wheelchair may also interfere with height measurement results. Ultrasonic methods also provide weak Identity information through height detection.

RFID sensing also has several drawbacks; it is difficult value proposition to get residential occupants to carry RFID tags with them continuously. Another drawback is that the triangulation methods discussed are too unreliable to place occupants in specific rooms in many cases.

WiFi association is not granular enough for residential use, as the original enterprise use case presupposed a much larger area, as well as multiple wireless access points, neither of which a typical residential environment have.

WiFi triangulation is a good candidate for residential use, as there are most likely neighbouring wireless networks that can be used as virtual landmarks. However, it suffers from the same granularity problems as WiFi association, as these signals are not specific enough to pinpoint an occupant to a specific room.

For approaches presupposing smartphones being present on each occupant, it is more difficult to ensure that occupants are carrying their smartphones with them at all times in a residential environment.  Another issue with smart phones is that they represent an expense that the target markets of the elderly and the disabled may not be able to afford.

Finally, we have thermal sensing. It provides both Presence and Count information, as it uses occupants' thermal signatures to determine the presence of people in a room. It does not however provide Identity information, as thermal signatures are not sufficiently unique with the technologies used to distinguished between occupants. Such a sensor system is presented as low-cost and energy efficient within Thermosense \cite{beltran2013thermosense}, is non-invasive by design and can reliably detect occupants with a very low root mean squared error. For our specific accessibility criteria, thermal sensing appears to be the best option available.

\section{Thermal sensors}
\label{sec:litreview:thermalsensors}
Our analysis (\Fref{subsec:litreview:sensors:analysis}) concluded that thermal sensors are the best candidates for this project. In this section we discuss the thermal sensing field in more detail.

A primary static/dynamic sensor fusion system in this field is the Thermosense system \cite{beltran2013thermosense}, a \pir and \iar\footnote{Phillips GridEYE; approx \$30} used to subdivide an area into an $8\times8$ grid of sections from which temperatures can be derived. This sensor system is attached to the roof on a small embedded controller which is responsible for collecting the data and transmitting it back to a larger computer via low powered wireless protocols.

The Thermosense system develops a thermal background map of the room using an \emwa over a 15 minute time window (if no motion is detected). If the room remains occupied for a long period, a more complex scaling algorithm is used which considers the coldest points in the room empty, and averages them against the new background, then performs \emwa with a lower weighting.

This background map is used as a baseline to calculate standard deviations of each grid area, which are then used to determine several characteristics to be used as feature vectors for a variety of classification approaches. The determination of the feature vectors was subject to experimentation, since the differences at each grid element too susceptible to individual room conditions to be used as feature vectors. Instead, a set of three different features was designed; the number of temperature anomalies in the space, the number of groups of temperature anomalies, and the size of the largest anomaly in the space. These feature vectors were compared against three classification approaches; K-Nearest Neighbors, Linear Regression and an a feed-forward Artificial Neural Network of one hidden later and 5 perceptions. All three classifiers achieved a Root Mean Squared Error (RMSE) within $0.38\pm0.04$. This final classification is subject to a final averaging process over a 4 minute window to remove the presence of independent errors from the raw classification data.

The Thermosense approach presents the state of the art in the field of sensing with \iar technology. Using a similar \iar system along with those types of classification algorithms should yield useful sensing results which can be then integrated into the broader sensor system.

\section{Research Gap}
\label{sec:litreview:discussion}
Throughout this review of the area of wireless occupancy sensors within the \iot it can be seen that there is a clear research gap within the area of occupancy. No group could be found who has assembled an occupancy sensor that optimises these ares of Low Cost, Non-Invasiveness, Energy Efficiency and Reliability into a architected software and hardware package that can be integrated like any other Thing into the \iot.

This is a key research area, because, as we have previously mentioned, the true ``disruptive level of innovation''\cite{atzori2010internet} the \iot provides can only be realised once a novel idea has been properly packaged as a Thing, rather than as a research curiosity. Packaging something as a Thing requires careful consideration of the best sensing systems, the best hardware to run those systems on, the best protocols to allow these Things to communicate, and the best device architecture to enable that communication. The state of the art in all these areas have been discussed throughout this literature review.

\section{Conclusion}
\label{sec:litreview:conclusion}
Several criteria were identified through which the spectrum of occupancy sensing could be examined; a quantitative criteria by Teixeira, Dublon and Savvides \cite{teixeira2010survey} to examine the different functionality offerings of sensor systems and a qualitative criteria derived from the aims of the project to examine how those sensors fit within the project's parameters.

Occupancy research performed with different sensor types was examined methodically through a set of taxonomic categories also originally proposed by Teixeira, Dublon and Savvides \cite{teixeira2010survey}, but modified to better suit the specifics of occupancy sensors. These sensor types included Thermal, \cdi, Video, Ultrasonic, \pir, RFID, various WiFi based methods, GPS and electricity consumption. Through an examination of these sensing systems quantitative and qualitative characteristics, it was determined that the Thermosense \iar system \cite{beltran2013thermosense} was the most suitable to the project's aims.

A key part of enabling the ``smart home for the disabled'' is creating a set of Things that can improve quality of life for those people. We believe our proposed Thing has clearly demonstrated this potential.

\ifcsdef{mainfile}{}{\bibliography{../references/primary}}
\end{document}