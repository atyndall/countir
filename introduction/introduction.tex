\documentclass[../thesis/thesis.tex]{subfiles}
\begin{document}
\chapter{Introduction}
The proportion of elderly and mobility-impaired people in the Australian population is predicted to grow dramatically over the next century, leaving a large proportion of the population unable to live independently~\cite{chan2009smart}. Human care for these groups is frequently performed on a volunteer basis, with nearly 40\% of such carers committing 40 hours or more per week on caring, and having significantly lower labour force participation as a result~\cite{abs4430}. Given the proportion of carers in the population is only projected to increase as the population ages, serious investments need to be made into technologies that can provide the support the elderly and mobility-imparted need to live independent of human assistance.

Additionally, the emergence of carbon pricing in many countries to combat anthropogenic climate change, as well as underinvestment in Australian energy infrastructure, are causing rising energy prices~\cite{energyprices}. These prices have a particularly large effect on the elderly and disabled, as these demographics typically have below-average incomes.

Coinciding with these issues is the booming embedded systems and sensor industries, which are creating increasingly smaller computer and sensing systems. Every iteration, these systems become more powerful, more affordable, and more networked. This phenomena, termed the \iot, has produced sub-\$50 devices such as the Arduino and Raspberry Pi which unlock enormous potential to create computing systems to help combat these and other issues. One can envision a future `smart home for the disabled' which leverages the \iot to offer a variety of services to help reduce financial and physical burdens alike.

Sensing techniques to determine occupancy, the detection of the presence and number of people in an area, are of particular use to the above demographics. Detection can be used to inform various devices to change state depending on the presence or absence of occupants, enabling a variety of useful automations. In particular, such a system could better regulate energy hungry devices to help reduce financial burden and greenhouse gas emissions. Household climate control, which in some regions of Australia accounts for up to 40\% of energy usage~\cite{abs4602} is one area in which occupancy detection can reduce costs. Several papers have found efficiency can be significantly increased, with some some approaches providing annual energy savings of up to 25\%~\cite{beltran2013thermosense}.
 
The field of sensors capable of predicting occupancy is broad. However, in many cases they suffer from problems of installation logistics, difficult assembly, assumptions on user's technology ownership and/or component cost. In the smart home envisioned, accuracy is important, but accessibility is paramount.

In this research project, we construct an \iot-style occupancy detection sensor system that forms part of a theoretical `smart home for the disabled.' This system must meet the following qualitative accessibility criteria;

\begin{itemize}
 \item \emph{Low Cost}: The set of components required should aim to minimise cost, as these devices are intended to be deployed in situations where the serviced user may be financially restricted. 

 We consider a prototype that costs less than \$300, with an anticipated reduction in cost as technology improves to be sufficiently low cost.
  
 \item \emph{Non-Invasive}: The sensors used in the system should gather as little information as necessary to achieve the detection goal; there are privacy concerns and adoption issues with sensors that are perceived to be invasive.

 We consider a sensor that obfuscates the identities and activities of those sensed to be sufficiently non-invasive.
 
 \item \emph{Reliable}: The system should be able to operate without user intervention or frequent maintenance, and should be able to perform its occupancy detection goal with a high degree of accuracy.

 We consider a system with an accuracy above 75\% to be sufficiently reliable for a prototype system.
 
 \item \emph{Energy Efficient}: The system may be placed in a location where there is no access to mains power, or where the retrofitting of appropriate power interfaces can be expensive (such as a residential roof); the ability to survive for long periods on only battery power is important.

 In the prototype stage, we consider a device that can run for one week on a battery that can be reasonably mounted on a roof to be sufficiently energy efficient.
\end{itemize}

Ultimately, this project attempts to answer the following research question: How can one create a complete sensing system to detect occupancy in a low cost, non-invasive, reliable and energy efficient way?

We take a traditional experimental approach to this research question by dividing the question into a series of sub-questions, resulting experiments and discussion. This occurs over three core chapters:

\begin{enumerate}
\item \emph{Literature Review}

An extensive literature review will be performed of the field of occupancy detection to determine which sensors types are most apparently appropriate for occupancy detection given the above criteria. 

The chapter identifies that existing systems do not meet our current needs and that low-resolution thermal sensing appears to offer the best maximization of our criteria.

\item \emph{Design}

Once the most appropriate sensor type is determined, a complete sensing system will be developed to provide a platform for experimentation and evaluation of the sensor, as well as to capture, store, visualize and replay sensor data for those purposes.

Firstly, we describe the development of an extensible hardware architecture based upon an \ard, Raspberry Pi, a \pir and the \mlx sensor.

Secondly, we describe the implementation of a custom and reusable software library, the \tarl, which consists of both low-level code running on the \ard embedded platform, and high-level code for image analysis and occupant prediction.

\item \emph{Evaluation}

Once a prototype exists, a methodology will be developed to evaluate the properties of the sensing system, and experiments will be designed and conducted to test different algorithms effectivenesses in using the system's data for occupancy detection.

Firstly, we investigate properties of the sensor that may influence the sensing system's ability to accurately detect occupants by performing a series of experiments and analysing the interesting properties of the resulting data.

Secondly, we detail the methodology by which thermal data and ground truth is captured, as well as the details of the software pipeline required to generate occupancy predictions. This includes how we approach machine learning, which algorithms we choose, and what parameters they use.

Thirdly, we perform a set of thermal captures, annotate them with ground truth information, then apply our chosen machine learning algorithms and occupancy detection methodology to the data to generate a series of results. We also measure the energy consumption of the prototype while capturing thermal data.

Finally, we compare our accuracy and energy efficiency data against the identified state of the art in the field, hypothesize as to why different machine learning approaches have achieved different results, and provide an in-depth analysis on the power consumption of the prototype and methods by which the energy efficiency could be further improved.
\end{enumerate}

\ifcsdef{mainfile}{}{\bibliography{../references/primary}}
\end{document}
