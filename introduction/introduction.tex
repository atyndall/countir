\documentclass[../thesis/thesis.tex]{subfiles}
\begin{document}
\chapter{Introduction}
The proportion of elderly and mobility-impaired people in the Australian population is predicted to grow dramatically over the next century, leaving a large proportion of the population unable to live independently~\cite{chan2009smart}. Human care for these groups is frequently performed on a volunteer basis, with nearly 40\% of such carers committing 40 hours or more per week on caring, and having significantly lower labor force participation as a result~\cite{abs4430}. Given the proportion of carers in the population is only projected to increase as the population ages, serious investments need to be made into technologies that can provide the support the elderly and mobility-imparted need to live independent of human assistance.

Additionally, the emergence of carbon pricing in many countries to combat anthropogenic climate change, as well as underinvestment in Australian energy infrastructure, are causing rising energy prices~\cite{energyprices}. These prices have a particularly large effect on the elderly and disabled, as these demographics typically have below-average incomes.

Coinciding with these issues is the booming embedded systems and sensor industries, which are creating increasingly smaller computer and sensing systems. Every iteration, these systems become more powerful, more affordable, and more networked. This phenomena, termed the \iot, has produced sub-\$50 devices such as the Arduino and Raspberry Pi which unlock enormous potential to create computing systems to help combat these and other issues. One can envision a future `smart home for the disabled' which leverages the \iot to offer a variety of services to help reduce financial and physical burdens alike.

Sensing techniques to determine occupancy, the detection of the presence and number of people in an area, are of particular use to the above demographics. Detection can be used to inform various devices to change state depending on the presence or absence of occupants, enabling a variety of useful automations. In particular, such a system could better regulate energy hungry devices to help reduce financial burden and greenhouse gas emissions. Household climate control, which in some regions of Australia accounts for up to 40\% of energy usage~\cite{abs4602} is one area in which occupancy detection can reduce costs. Several papers have found efficiency can be significantly increased, with some some approaches providing annual energy savings of up to 25\%~\cite{beltran2013thermosense}.
 
The field of sensors capable of predicting occupancy is quite broad. However, in many cases they suffer from problems of installation logistics, difficult assembly, assumptions on user's technology ownership and/or component cost. In the smart home envisioned, accuracy is important, but accessibility is paramount.

In this research project, we construct an \iot-style occupancy detection sensor system that forms part of a theoretical `smart home for the disabled.' This system must meet the following qualitative accessibility criteria;

\begin{itemize}
 \item \emph{Low Cost}: The set of components required should aim to minimise cost, as these devices are intended to be deployed in situations where the serviced user may be financially restricted.
 
 \item \emph{Non-Invasive}: The sensors used in the system should gather as little information as necessary to achieve the detection goal; there are privacy concerns and adoption issues with sensors that are perceived to be invasive.
 
 \item \emph{Reliable}: The system should be able to operate without user intervention or frequent maintenance, and should be able to perform its occupancy detection goal with a high degree of accuracy.
 
 \item \emph{Energy Efficient}: The system may be placed in a location where there is no access to mains power, or where the retrofitting of appropriate power interfaces can be expensive (such as a residential roof); the ability to survive for long periods on only battery power is advantageous.
\end{itemize}

Constructing a sensor system that meets these criteria involves a four step process outlined below. Each of these steps is dedicated a chapter in this dissertation.
\begin{enumerate}
\item \emph{Literature Review}

An extensive literature review will be performed of the field of occupancy detection to determine which sensors types are most apparently appropriate for occupancy detection given the above criteria.

We identify that existing systems do not meet our current needs, and identify which types of sensors would be well suited to our occupancy detection task, concluding that thermal sensing is the most promising area.

\item \emph{Design}

Once the most appropriate sensor type is determined, a hardware and software prototype will be developed to provide a platform for experimentation and evaluation of the sensor, as well as to capture, store, visualize and replay sensor data for those purposes.

We take our design and build a fully functioning hardware prototype capable of capturing thermal, visual and motion data at speeds of up to 8~Hz. This sensing system is based upon an \ard, Raspberry Pi, a \pir and the \mlx sensor.

\item \emph{Evaluation}

Once a prototype exists, a methodology will be developed to evaluate the properties of the sensing system, and experiments will be designed and conducted to test different algorithms effectivenesses in using the system's data for occupancy detection.

We experiment extensively with our hardware prototype, and discover interesting properties relating to bias, sensitivity and noise. We orchestrate a series of scripted scenarios and record them with our prototype, and use custom built software to efficiently tag and aggregate the corresponding data.

Using the collected data, we delve into machine learning algorithms, and evaluate a suite of machine learning algorithms to test the prototype's occupancy detection accuracy. Our results are intriguing, highlighting the important role of entropy in accurate classification of our dataset.

\item \emph{Conclusions}

The prototype's experimental results will be analyzed to conclude as to its effectiveness with relation to the above criteria. Research limitations and future work are also discussed.

We evaluate how our sensing system meets our original goals and conclude that the project has been ultimately successful in achieving them.
\end{enumerate}

% ASK: Right way to go about telling people what is going to happen?
 
\ifcsdef{mainfile}{}{\bibliography{../references/primary}}
\end{document}
