\documentclass[../thesis/thesis.tex]{subfiles}
\begin{document}
\chapter{Introduction}
The proportion of elderly and mobility-impaired people in the Australian population is predicted to grow dramatically over the next century, leaving a large proportion of the population unable to live independently~\cite{chan2009smart}. Human care for these groups is often performed on a volunteer basis, with nearly 40\% of such carers committing 40 hours or more per week on caring. These carers have significantly lower labour force participation as a result~\cite{abs4430}. Given the proportion of carers in the population is projected to increase as the population ages, investment needs to be made into technologies that can reduce the burden on these carers.

Additionally, the emergence of carbon pricing in many countries to combat anthropogenic climate change is causing rising energy prices generally, while underinvestment in Australian energy infrastructure is causing rising energy prices in Australia specifically~\cite{energyprices}. These rising prices have a particularly large effect on the elderly and disabled, as these demographics typically have below-average incomes. These prices are also forcing medium-to-large office-based businesses to consider means by which to reduce their overall power consumption.

Coinciding with these issues is the development of increasing smaller computing and sensing systems, which provide a potential solution to these rising costs. At every iteration, these systems become more powerful, more affordable, and more networked. This phenomena, termed the \iot, has produced sub-\$50 embedded computing devices such as the Arduino and Raspberry Pi. These systems unlock enormous potential to create computing solutions to these rising prices and other important issues. One can envision a future `smart home' or `smart workplace' which leverages the \iot to offer a variety of services to help reduce financial and physical burdens alike.

Sensing techniques to determine occupancy, the detection of the presence and number of people in an area, are of particular use to the both workplaces and residences alike. Detection can be used to inform various devices to change state depending on the presence or absence of occupants, enabling a variety of useful automations. In particular, such a detection system could better regulate energy intensive devices to help reduce financial burden and greenhouse gas emissions. Household climate control, which in some regions of Australia accounts for up to 40\% of energy usage~\cite{abs4602} is one area in which occupancy detection can reduce costs. More finely grained control of an office building's Heating, Ventilating, and Air Conditioning (HVAC) system is another potential application. Several papers have found climate control efficiency can be significantly increased, with some approaches providing annual energy savings of up to 25\%~\cite{beltran2013thermosense}.
 
Occupancy sensors are broadly characterised. In many cases existing sensors suffer from problems of installation logistics, difficult assembly, assumptions on user's technology ownership and/or component cost. In the smart home/workplace envisioned, accuracy is important, but accessibility is paramount.

In this research project, we construct an \iot-style occupancy detection sensor system that forms part of a smart home or workplace. This system must meet the following criteria:

\begin{itemize}
 \item \emph{Low Cost}: The set of components required should minimise cost, as these devices are intended to be deployed in situations where the serviced user may be financially restricted, or where many units would be required.

 We consider a prototype that costs less than \$300, with an anticipated reduction in cost as technology improves to be sufficiently low cost.
  
 \item \emph{Non-Invasive}: The sensors used in the system should gather as little information as necessary to achieve the detection goal; there are privacy concerns and adoption issues with sensors that are perceived to be invasive.

 We consider a sensor that significantly obfuscates the identities and activities of those sensed to be sufficiently non-invasive.
 
 \item \emph{Reliable}: The system should be able to operate without user intervention or frequent maintenance, and should be able to perform its occupancy detection goal with a high degree of accuracy.

 We consider a system with an accuracy above 75\% to be sufficiently reliable for a prototype system.
 
 \item \emph{Energy Efficient}: The system may be placed in a location where there is no access to mains power, or where the retrofitting of appropriate power interfaces can be expensive (such as a residential roof); the ability to survive for long periods on only battery power is important.

 In the prototype stage, we consider a device that can run for one week on a battery that can be reasonably mounted on a roof to be sufficiently energy efficient.
\end{itemize}

Ultimately, this project attempts to answer the following research question: How can one create a complete sensing system to detect occupancy in a low cost and non-invasive way, with additional considerations to reliability and energy efficiency?

We pursued an experimental approach to this broad research question by dividing the answer into a series of different investigations, design processes, resulting experiments and discussions. This occurred over a period of 12~months, and is reviewed over three core chapters:

% TODO: Tenses here
\begin{enumerate}
\item \emph{Literature Review}

An extensive literature review was performed of the field of occupancy detection to determine which sensors types were most apparently appropriate for occupancy detection given the above criteria. 

In the chapter we identified the state of the art in various sensing categories, we then evaluated them against both a set of qualitative and quantitative criteria, and we then determined that low-resolution thermal sensing presents the best accuracy/non-invasiveness trade-off.

\item \emph{Design}

With the appropriate sensor type then determined, a complete sensing system was developed to provide a platform for experimentation and evaluation of the sensor, as well as to capture, store, visualize and replay sensor data for those purposes.

Firstly, we describe the development of an extensible hardware architecture based upon an \ard, Raspberry Pi, a \pir and the \mlx sensor.

Secondly, we describe the implementation of a custom and reusable software library, \tarl, which consists of both low-level code running on the \ard embedded platform, and high-level code for image analysis and occupant prediction running on the Raspberry Pi.

\item \emph{Evaluation}

With the creation of the prototype, a methodology was then developed to evaluate the properties of the sensing system, and experiments were designed and conducted to test different algorithms' effectivenesses in using the system's data for occupancy detection.

Firstly, we investigate properties of the sensor that may influence the sensing system's ability to accurately detect occupants by performing a series of experiments and analysing the interesting properties of the resulting data.

Secondly, we detail the methodology by which thermal data and ground truth is captured, as well as the details of the software pipeline required to generate occupancy predictions. This includes how we approach machine learning, which algorithms we choose, and what parameters they use.

Thirdly, we perform a set of thermal captures, annotate them with ground truth information, then apply our chosen machine learning algorithms and occupancy detection methodology to the data to generate a series of results. We also measure the energy consumption of the prototype while capturing thermal data.

Finally, we compare our accuracy and energy efficiency data against the identified state of the art in the field, hypothesize as to why different machine learning approaches have achieved different results, and provide an in-depth analysis on the power consumption of the prototype and methods by which the energy efficiency could be further improved.
\end{enumerate}

\ifcsdef{mainfile}{}{\bibliography{../references/primary}}
\end{document}
